%% Some useful variations of the document class
%%\documentclass[draft]{beamer}
%%\documentclass[handout]{beamer}
% silence hyperref warnings
\documentclass[hyperref={pdfpagelabels=false}]{beamer}
%\documentclass{beamer}
\usepackage[T1]{fontenc}
\usepackage[utf8]{inputenc}
\usepackage[ngerman]{babel}
\usepackage{wasysym}
\usepackage{ziffer}
\mode<presentation>{\usetheme{Copenhagen}}
\title{A CamlLight Compiler for the JVM}
\author{Hupel, Kubica, Müller, Schulze, Traytel}
\institute{TU München}
\date{20.~August~2010}

%% get rid of navigation symbols
%\setbeamertemplate{navigation symbols}{}
\beamertemplatenavigationsymbolsempty

\begin{document}

\frame{\titlepage}

% define the logo after the title page
\logo{\includegraphics[width=1.5cm]{tun}}

\begin{frame}{Lexer}
  \begin{itemize}
    \item JFlex
  \end{itemize}
\end{frame}

\begin{frame}{Parser}
  \begin{itemize}
    \item CUP2
    \item CamlLight"=Syntax
  \end{itemize}
\end{frame}

\begin{frame}{Typinferenz}
\end{frame}

\begin{frame}{Codegenerierung: MaMa}
\end{frame}

\begin{frame}{Codegenerierung: JVM"=Bytecode}
  \begin{block}{Idee}
    \begin{itemize}
      \item Verarbeitet erweiterten MaMa"=Code: deineMaMa
        \begin{itemize}
          \item Instruktionen die PC lesen, nehmen stattdessen Label als Parameter
          \item Instruktionen die PC modifizieren, geben stattdessen Label aus
        \end{itemize}
      \item Verwendung des JVM"=Heaps
      \item Verwendung von \texttt{java.util.Stack}
      \item Instruktionen in Java geschrieben
    \end{itemize}
  \end{block}
  \begin{block}{ASM}
    \begin{itemize}
      \item Library für Bytecodeinspektion und Modifikation
      \item Wird genutzt um unsere eigene \texttt{main()} einzuschleusen
    \end{itemize}
  \end{block}
\end{frame}

\begin{frame}{Zustand}
  \begin{exampleblock}{Was funktioniert}
    \begin{itemize}
      \item Syntax, mit (sinnvollen?) Fehlermeldungen
      \item Lambdas, Match, Tupel, Records, Strings, Listen
      \item Ausgabe von Java"=Bytecode
    \end{itemize}
  \end{exampleblock}
  \begin{alertblock}{Was nicht funktioniert}
    \begin{itemize}
      \item Typkonstruktoren
      \item Seiteneffekte, Exceptions
      \item Threads
    \end{itemize}
  \end{alertblock}
\end{frame}

\begin{frame}{Benchmarks}
  \begin{block}{Ackermann(3, 5)}
    \begin{tabular}{l|l|l}
       & Scala & TUM \\
      Kompilieren & 9.3s & 4.1s \\
      Ausführen & 1.03s & 1.69s \\
    \end{tabular}
  \end{block}
  \begin{block}{Factorial(10)}
    \begin{tabular}{l|l|l}
       & Scala & TUM \\
      Kompilieren & 9.1s & 3.9s \\
      Ausführen & 0.98s & 0.13s \\
    \end{tabular}
  \end{block}
\end{frame}

\begin{frame}{Stats}
  \begin{block}{Code}
    $\approx 3.300$ LOC Scala, $650$ LOC Java, $> 250$ Commits
  \end{block}
  \begin{block}{Best of Commit"=Messages}
    \begin{itemize}
      \item Foo
    \end{itemize}
  \end{block}
\end{frame}

\begin{frame}{Rückblick}
  \begin{exampleblock}{Was war gut}
    \begin{itemize}
      \item Scala Patternmatching
      \item Betreuung \smiley
    \end{itemize}
  \end{exampleblock}
  \begin{alertblock}{Was war weniger gut}
    \begin{itemize}
      \item NetBeans
      \item Compilezeiten von \texttt{scalac}
    \end{itemize}
  \end{alertblock}
\end{frame}

\begin{frame}{Zukunft}
\end{frame}

\end{document}
